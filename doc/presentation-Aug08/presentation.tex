\documentclass{beamer}

% This file is a solution template for:

% - Talk at a conference/colloquium.
% - Talk length is about 20min.
% - Style is ornate.



% Copyright 2004 by Till Tantau <tantau@users.sourceforge.net>.
%
% In principle, this file can be redistributed and/or modified under
% the terms of the GNU Public License, version 2.
%
% However, this file is supposed to be a template to be modified
% for your own needs. For this reason, if you use this file as a
% template and not specifically distribute it as part of a another
% package/program, I grant the extra permission to freely copy and
% modify this file as you see fit and even to delete this copyright
% notice. 


\mode<presentation>
{
  \usetheme{Madrid}
  % or ...

  \setbeamercovered{transparent}
  % or whatever (possibly just delete it)
}


\usepackage{listings}
\usepackage[english]{babel}

% or whatever

\usepackage[latin1]{inputenc}
% or whatever

\usepackage{times}
\usepackage[T1]{fontenc}
% Or whatever. Note that the encoding and the font should match. If T1
% does not look nice, try deleting the line with the fontenc.


%signals, used in Ingo's figures
\newcommand{\signal}[1]{$\overrightarrow{#1}$}

\graphicspath{{../report/figures/}{figures/}}

% grey for the listings


% settings for the listings
\lstset{language=Haskell,
  linewidth=.9\linewidth,
  stringstyle=\ttfamily,
  basicstyle=\scriptsize\ttfamily,
  frame=lines,
  frameround=ffff,
  backgroundcolor=\color[rgb]{.9,.9,1}}


\title%[Short Paper Title]  (optional, use only with long paper titles)
{ForSyDe's embedded compiler}

\subtitle{Third development stage results.}

\author[A.Acosta] % (optional, use only with lots of authors)
{Alfonso Acosta\\
\footnotesize \href{mailto:alfonso.acosta@gmail.com}{\nolinkurl{alfonso.acosta@gmail.com}}}

% - Give the names in the same order as the appear in the paper.
% - Use the \inst{?} command only if the authors have different
%   affiliation.

\institute[KTH] % (optional, but mostly needed)
{ICT/ECS\\Royal Institute of Technology, Stockholm}
% - Use the \inst command only if there are several affiliations.
% - Keep it simple, no one is interested in your street address.

\date%[CFP 2003] % (optional, should be abbreviation of conference name)
{August 18th, 2008}
% - Either use conference name or its abbreviation.
% - Not really informative to the audience, more for people (including
%   yourself) who are reading the slides online

\subject{Compilers}
% This is only inserted into the PDF information catalog. Can be left
% out. 



% If you have a file called "university-logo-filename.xxx", where xxx
% is a graphic format that can be processed by latex or pdflatex,
% resp., then you can add a logo as follows:

\pgfdeclareimage[height=0.5cm]{university-logo}{kth_cmyk}
\logo{\pgfuseimage{university-logo}}



% Delete this, if you do not want the table of contents to pop up at
% the beginning of each subsection:
\AtBeginSubsection[]
{
  \begin{frame}<beamer>
    \frametitle{Outline}
    \tableofcontents[currentsection,currentsubsection]
  \end{frame}
}

\AtBeginSection[]
{
  \begin{frame}<beamer>
    \frametitle{Outline}
    \tableofcontents[currentsection]
  \end{frame}
}


% If you wish to uncover everything in a step-wise fashion, uncomment
% the following command: 

\beamerdefaultoverlayspecification{<+->}


\begin{document}

\begin{frame}
  \titlepage
\end{frame}

\begin{frame}
  \frametitle{Outline}
  \tableofcontents[pausesections]
  % You might wish to add the option [pausesections]
\end{frame}


% Structuring a talk is a difficult task and the following structure
% may not be suitable. Here are some rules that apply for this
% solution: 

% - Exactly two or three sections (other than the summary).
% - At *most* three subsections per section.
% - Talk about 30s to 2min per frame. So there should be between about
%   15 and 30 frames, all told.

% - A conference audience is likely to know very little of what you
%   are going to talk about. So *simplify*!
% - In a 20min talk, getting the main ideas across is hard
%   enough. Leave out details, even if it means being less precise than
%   you think necessary.
% - If you omit details that are vital to the proof/implementation,
%   just say so once. Everybody will be happy with that.

\beamerdefaultoverlayspecification{}
\section{General Goal Review}
\begin{frame}
  \frametitle{General Goal Review}
  %\framesubtitle{Subtitles are optional.}
  % - A title should summarize the slide in an understandable fashion
  %   for anyone how does not follow everything on the slide itself.
  Plan set after last meeting (April 25th)
  \begin{itemize}
  \item<1-> July 1st.
    \begin{itemize}
    \item Automatic Test Suite with code
      coverage. \visible<2->{\pgfimage[height=10pt]{figures/tick}
        \alert{\small No CC due to TH.}}
    \item Stable VHDL Backend. \visible<2->{\pgfimage[height=10pt]{figures/tick}}
    \item Improve GraphML backend. \visible<2->{\pgfimage[height=10pt]{figures/tick}}
    \item Additional examples. \visible<2->{\pgfimage[height=10pt]{figures/tick}}
    \end{itemize}
  \item<3-> August 18th to August 24th: 3rd meeting at KTH.
    
  \item<4-> September 1st: Public release
    \begin{itemize}
      \item Full tutorial. \visible<5->{\pgfimage[height=10pt]{figures/tick}}
      \item Hacking Guide. \visible<5->{\pgfimage[height=10pt]{figures/tick}}
      \item Even more code examples. \visible<5->{\pgfimage[height=10pt]{figures/cross}}
      \item Implement Domain interfaces and other MoCs (Untimed,
        optionally CT), either
        .. \visible<5->{\pgfimage[height=10pt]{figures/tick} \alert{Option 2
          (Swallow-embedded \texttt{Signal})}}
        \begin{enumerate}
        \item porting them to the new deep-embedded \texttt{Signal} respresentation
        \item interfacing with the old swallow-embedded \texttt{Signal}.
        \end{enumerate}
    \end{itemize}
  \end{itemize}
\end{frame}

\section{Result details}

\beamerdefaultoverlayspecification{<+->}



\subsection{Frontend}

\begin{frame}[fragile]
  \frametitle{Reminder: Design Flow Using Components (I)}
  %\framesubtitle{Components.}
  % - A title should summarize the slide in an understandable fashion
  %   for anyone how does not follow everything on the slide itself.
  \begin{itemize}
  \item
  Naive example of the design flow and API using components: Serial adder.
  \visible<2->{\pgfimage[width=10cm]{figures/SeqAddFour}}
  \end{itemize}
  
  
  \vspace{-0.7cm}
  \begin{overprint}
   \onslide<3>
   \begin{enumerate}[1)]
   \item Create a process function which adds one to its input 
    \begin{lstlisting}
  addOnef :: ProcFun (Int -> Int)
  addOnef = $(newProcFun [d| addOnef :: Int -> Int 
                             addOnef n = n + 1 |]
    \end{lstlisting}
    \end{enumerate}
   
   \onslide<4>
   \begin{enumerate}[2)]
   \item Create a system function corresponding to the unit adder
   \begin{lstlisting}
  addOneProc :: Signal Int -> Signal Int
  addOneProc = mapSY "addOne" addOnef
   \end{lstlisting}
   \end{enumerate}
   
   \onslide<5>
   \begin{enumerate}[3)]
   \item Subsystem definition associated to the unit adder
   \begin{lstlisting}
 addOneSysDef :: SysDef (Signal Int -> Signal Int)
 addOneSysDef = $(newSysDef 'addOneProc ["in1"] ["out1"])
   \end{lstlisting}
   \end{enumerate}


   \onslide<6>
   \begin{enumerate}[4)]
   \item Create the main system function
   \begin{lstlisting}
  addFour :: Signal Int -> Signal Int
  addFour = $(instantiate "addOne3" 'addOneSysDef) .
            $(instantiate "addOne2" 'addOneSysDef) .
            $(instantiate "addOne1" 'addOneSysDef) .
            $(instantiate "addOne0" 'addOneSysDef)
   \end{lstlisting}
   \end{enumerate}

   \onslide<7>
   \begin{enumerate}[5)]
   \item Finally, build the main system definition
   \begin{lstlisting}
  addFourSys :: SysDef (Signal Int -> Signal Int)
  addFourSys = $(newSysDef 'addFour ["in1"] ["out1"])
   \end{lstlisting}
   \end{enumerate}

\end{overprint}

\end{frame}

\begin{frame}
  \frametitle{Reminder: Design Flow Using Components (II)}
  %\framesubtitle{Design flow using components.}
  % - A title should summarize the slide in an understandable fashion
  %   for anyone how does not follow everything on the slide itself.
\vspace{-0.2cm}
\begin{center}
\pgfimage[height=8cm]{figures/compflow}
\end{center}

\end{frame}



\begin{frame}[fragile]
  \frametitle{Frontend Changes: Keep TH to a minimum. (I)}
  %\framesubtitle{Components.}
  % - A title should summarize the slide in an understandable fashion
  %   for anyone how does not follow everything on the slide itself.
  \begin{itemize}
  \item<1-> The frontend functions \texttt{newSysDef},
    \texttt{instantiate} and \texttt{simulate} don't depend on
    Template Haskell anymore.

\begin{lstlisting}
newSysDef :: Name -> [PortId] -> [PortId] -> Q Exp
 -- sysfun -> [PortId] -> [PortId] -> SysDef (sysfun)
instantiate :: Name -> ProcId -> Q Exp -- SysDef a -> a
simulate :: Name -> Q Exp -- SysDef a -> listbased a 
\end{lstlisting}
\begin{lstlisting}
newSysDef :: (SysFun f) => f -> SysId -> 
                           [PortId] -> [PortId] ->
                           SysDef f
instantiate :: (SysFun f) => ProcId -> SysDef f -> f
simulate :: (SysFunToSimFun sysFun simFun) => 
            SysDef sysFun -> simFun
\end{lstlisting}

\item<2->     
  \begin{itemize}
  \item Saves the user from the inconvenient TH syntax.
  \item System patterns involving instantiate are now possible
  \item Does not suffer the type-inference problems of TH.
  \item Typing of frontend functions becomes more intuitive
  \item \alert{drawback}: compile-time checks are lost (\texttt{newSysDefTH}).
  \end{itemize}
\end{itemize}
\end{frame}


\begin{frame}[fragile]
  \frametitle{Frontend Changes: Keep TH to a minimum. (II)}
  %\framesubtitle{Components.}
  % - A title should summarize the slide in an understandable fashion
  %   for anyone how does not follow everything on the slide itself.
  \begin{itemize}
  \item<1-> System functions have an unrestricted and variable number of
    argument. Is it possible to describe with type classes
    (\texttt{SysFun} and \texttt{SysFunToSimFun})?
  \item Yes! The solution is inspired in a similar, previously solved
    problem (implementing printf for Haskell).
  \item Omitting 

\begin{lstlisting}
newSysDef :: Name -> [PortId] -> [PortId] -> Q Exp
 -- sysfun -> [PortId] -> [PortId] -> SysDef (sysfun)
instantiate :: Name -> ProcId -> Q Exp -- SysDef a -> a
simulate :: Name -> Q Exp -- SysDef a -> listbased a 
\end{lstlisting}
\begin{lstlisting}
newSysDef :: (SysFun f) => f -> SysId -> 
                           [PortId] -> [PortId] ->
                           SysDef f
instantiate :: (SysFun f) => ProcId -> SysDef f -> f
simulate :: (SysFunToSimFun sysFun simFun) => 
            SysDef sysFun -> simFun
\end{lstlisting}

\item<2->     
  \begin{itemize}
  \item Saves the user from the inconvenient TH syntax.
  \item System patterns involving instantiate are now possible
  \item Does not suffer the type-inference problems of TH.
  \item Typing of frontend functions becomes more intuitive
  \item \alert{drawback}: compile-time checks are lost (\texttt{newSysDefTH}).
  \end{itemize}
\end{itemize}
\end{frame}


\subsection{GraphML backend}

\begin{frame}[fragile]
  \frametitle{GraphML backend}
  %\framesubtitle{Components.}
  % - A title should summarize the slide in an understandable fashion
  %   for anyone how does not follow everything on the slide itself.
  \begin{itemize}
  \item lala
  \end{itemize}
  
\end{frame}


\subsection{VHDL backend}

\begin{frame}[fragile]
  \frametitle{VHDL backend}
  %\framesubtitle{Components.}
  % - A title should summarize the slide in an understandable fashion
  %   for anyone how does not follow everything on the slide itself.
  \begin{itemize}
  \item lala
  \end{itemize}
  
\end{frame}

\subsection{Testing}

\begin{frame}
  \frametitle{Testing}
  %\framesubtitle{Components.}
  % - A title should summarize the slide in an understandable fashion
  %   for anyone how does not follow everything on the slide itself.
  \begin{itemize}
  \item lala
  \end{itemize}
  
\end{frame}

\subsection{New, shallow-embedded MoCs}

\begin{frame}
  \frametitle{New, shallow-embedded MoCs}
  %\framesubtitle{Components.}
  % - A title should summarize the slide in an understandable fashion
  %   for anyone how does not follow everything on the slide itself.
  \begin{itemize}
  \item lala
  \end{itemize}
  
\end{frame}

\subsection{Documentation}

\begin{frame}
  \frametitle{Documentation}
  %\framesubtitle{Components.}
  % - A title should summarize the slide in an understandable fashion
  %   for anyone how does not follow everything on the slide itself.
  \begin{itemize}
  \item Web page and tutorial.
  \end{itemize}
  
\end{frame}

\section{Lessons Learned}

\section{Specific topics to discuss}

\section{What's next?}

\begin{frame}
  \frametitle{What's next?}
  %\framesubtitle{Design flow using components.}
  % - A title should summarize the slide in an understandable fashion
 \begin{itemize}
 \item What should be done now?
 \end{itemize}
\end{frame}

\end{document}


