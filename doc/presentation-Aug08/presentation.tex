\documentclass{beamer}

% This file is a solution template for:

% - Talk at a conference/colloquium.
% - Talk length is about 20min.
% - Style is ornate.



% Copyright 2004 by Till Tantau <tantau@users.sourceforge.net>.
%
% In principle, this file can be redistributed and/or modified under
% the terms of the GNU Public License, version 2.
%
% However, this file is supposed to be a template to be modified
% for your own needs. For this reason, if you use this file as a
% template and not specifically distribute it as part of a another
% package/program, I grant the extra permission to freely copy and
% modify this file as you see fit and even to delete this copyright
% notice. 


\mode<presentation>
{
  \usetheme{Madrid}
  % or ...

  \setbeamercovered{transparent}
  % or whatever (possibly just delete it)
}


\usepackage{listings}
\usepackage[english]{babel}

% or whatever

\usepackage[latin1]{inputenc}
% or whatever

\usepackage{times}
\usepackage[T1]{fontenc}
% Or whatever. Note that the encoding and the font should match. If T1
% does not look nice, try deleting the line with the fontenc.


%signals, used in Ingo's figures
\newcommand{\signal}[1]{$\overrightarrow{#1}$}

\graphicspath{{../report/figures/}{figures/}}

% grey for the listings


% settings for the listings
\lstset{language=Haskell,
  linewidth=.9\linewidth,
  stringstyle=\ttfamily,
  basicstyle=\scriptsize\ttfamily,
  frame=lines,
  frameround=ffff,
  backgroundcolor=\color[rgb]{.9,.9,1}}


\title%[Short Paper Title]  (optional, use only with long paper titles)
{ForSyDe's embedded compiler}

\subtitle{Third development stage results.}

\author[A.Acosta] % (optional, use only with lots of authors)
{Alfonso Acosta\\
\footnotesize \href{mailto:alfonso.acosta@gmail.com}{\nolinkurl{alfonso.acosta@gmail.com}}}

% - Give the names in the same order as the appear in the paper.
% - Use the \inst{?} command only if the authors have different
%   affiliation.

\institute[KTH] % (optional, but mostly needed)
{ICT/ECS\\Royal Institute of Technology, Stockholm}
% - Use the \inst command only if there are several affiliations.
% - Keep it simple, no one is interested in your street address.

\date%[CFP 2003] % (optional, should be abbreviation of conference name)
{August 18th, 2008}
% - Either use conference name or its abbreviation.
% - Not really informative to the audience, more for people (including
%   yourself) who are reading the slides online

\subject{Compilers}
% This is only inserted into the PDF information catalog. Can be left
% out. 



% If you have a file called "university-logo-filename.xxx", where xxx
% is a graphic format that can be processed by latex or pdflatex,
% resp., then you can add a logo as follows:

\pgfdeclareimage[height=0.5cm]{university-logo}{kth_cmyk}
\logo{\pgfuseimage{university-logo}}



% Delete this, if you do not want the table of contents to pop up at
% the beginning of each subsection:
\AtBeginSubsection[]
{
  \begin{frame}<beamer>
    \frametitle{Outline}
    \tableofcontents[currentsection,currentsubsection]
  \end{frame}
}

\AtBeginSection[]
{
  \begin{frame}<beamer>
    \frametitle{Outline}
    \tableofcontents[currentsection]
  \end{frame}
}


% If you wish to uncover everything in a step-wise fashion, uncomment
% the following command: 

\beamerdefaultoverlayspecification{<+->}


\begin{document}

\begin{frame}
  \titlepage
\end{frame}

\begin{frame}
  \frametitle{Outline}
  \tableofcontents[pausesections]
  % You might wish to add the option [pausesections]
\end{frame}


% Structuring a talk is a difficult task and the following structure
% may not be suitable. Here are some rules that apply for this
% solution: 

% - Exactly two or three sections (other than the summary).
% - At *most* three subsections per section.
% - Talk about 30s to 2min per frame. So there should be between about
%   15 and 30 frames, all told.

% - A conference audience is likely to know very little of what you
%   are going to talk about. So *simplify*!
% - In a 20min talk, getting the main ideas across is hard
%   enough. Leave out details, even if it means being less precise than
%   you think necessary.
% - If you omit details that are vital to the proof/implementation,
%   just say so once. Everybody will be happy with that.

\beamerdefaultoverlayspecification{}
\section{General Goal Review}
\begin{frame}
  \frametitle{General Goal Review}
  %\framesubtitle{Subtitles are optional.}
  % - A title should summarize the slide in an understandable fashion
  %   for anyone how does not follow everything on the slide itself.
  \begin{itemize}
  \item
    During this first stage it was agreed to obtain a robust implementation attaining to this particular goals:
    \begin{itemize}
    \item<2-> Finish the implementation of components (previously named Blocks and Ports).  \visible<3->{\pgfimage[height=10pt]{figures/tick}}
    \item<4-> Add a simulation backend with support for \textbf{any} signal type.\visible<5->{\pgfimage[height=10pt]{figures/tick}}
    \item<6-> Support all the synchronous process constructors in ForSyDe.\visible<7->{\pgfimage[height=10pt]{figures/tick}}
    \item<8-> Improve the error handling and reporting of the compiler.\visible<9->{\pgfimage[height=10pt]{figures/tick}}
    \item<10-> Optionally. Document the code with haddock and cabalize the project.\visible<11->{\pgfimage[height=10pt]{figures/tick}}
    \item<12-> Create a project webpage\visible<13->{\pgfimage[height=10pt]{figures/cross} Not until there's a release}
    \item<14-> Improve the VHDL backend.\visible<15->{\pgfimage[height=10pt]{figures/cross} Still not advanced support}
    \end{itemize}
 \item<16-> Additional work done
 	\begin{itemize}
	\item The project was rewritten from scratch (slightly based on the old code).
		\begin{itemize}
		\item<17-> New module hierarchy\visible<17->{\pgfimage[height=8pt]{figures/tick}}
		\item<18-> More general identifiers.\visible<18->{\pgfimage[height=8pt]{figures/tick}}
		\item<19-> 	Access to external scope of \texttt{ProcFun}s.\visible<19->{\pgfimage[height=8pt]{figures/tick}}
		\item<20-> Redesigned component API.\visible<20->{\pgfimage[height=8pt]{figures/tick}}
		\end{itemize}
	\end{itemize}
  \end{itemize}
\end{frame}

\section{Goal details}

\beamerdefaultoverlayspecification{<+->}

\subsection{Frontend}

\begin{frame}[fragile]
  \frametitle{Frontend}
  %\framesubtitle{Components.}
  % - A title should summarize the slide in an understandable fashion
  %   for anyone how does not follow everything on the slide itself.
  \begin{itemize}
  \item lala
  \end{itemize}
  
\end{frame}



\subsection{Frontend}

\begin{frame}[fragile]
  \frametitle{Frontend}
  %\framesubtitle{Components.}
  % - A title should summarize the slide in an understandable fashion
  %   for anyone how does not follow everything on the slide itself.
  \begin{itemize}
  \item lala
  \end{itemize}
  
\end{frame}


\subsection{GraphML backend}

\begin{frame}[fragile]
  \frametitle{GraphML backend}
  %\framesubtitle{Components.}
  % - A title should summarize the slide in an understandable fashion
  %   for anyone how does not follow everything on the slide itself.
  \begin{itemize}
  \item lala
  \end{itemize}
  
\end{frame}


\subsection{VHDL backend}

\begin{frame}[fragile]
  \frametitle{VHDL backend}
  %\framesubtitle{Components.}
  % - A title should summarize the slide in an understandable fashion
  %   for anyone how does not follow everything on the slide itself.
  \begin{itemize}
  \item lala
  \end{itemize}
  
\end{frame}

\subsection{Testing}

\begin{frame}
  \frametitle{Testing}
  %\framesubtitle{Components.}
  % - A title should summarize the slide in an understandable fashion
  %   for anyone how does not follow everything on the slide itself.
  \begin{itemize}
  \item lala
  \end{itemize}
  
\end{frame}

\subsection{New, shallow-embedded MoCs}

\begin{frame}
  \frametitle{New, shallow-embedded MoCs}
  %\framesubtitle{Components.}
  % - A title should summarize the slide in an understandable fashion
  %   for anyone how does not follow everything on the slide itself.
  \begin{itemize}
  \item lala
  \end{itemize}
  
\end{frame}

\subsection{Documentation}

\begin{frame}
  \frametitle{Documentation}
  %\framesubtitle{Components.}
  % - A title should summarize the slide in an understandable fashion
  %   for anyone how does not follow everything on the slide itself.
  \begin{itemize}
  \item Web page and tutorial.
  \end{itemize}
  
\end{frame}

\section{Lessons Learned}

\section{Specific topics to discuss}

\section{What's next?}

\begin{frame}
  \frametitle{What's next?}
  %\framesubtitle{Design flow using components.}
  % - A title should summarize the slide in an understandable fashion
 \begin{itemize}
 \item What should be done now?
 \end{itemize}
\end{frame}

\end{document}


