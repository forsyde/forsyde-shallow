
The ForSyDe (\textit{Formal System Design}) methodology
is targeted at modelling systems, with the goal of using a high level
of abstraction in the specification of its models.

Although it is a general system modelling methodology, the initial
scope of ForSyDe has specifically been \textit{Synchronous Systems}
(systems in which a global clock is used to synchronize the different
parts of the system). A well-known type of such system is synchronous
hardware, which is the main subject of this thesis. A synchronous
system in ForSyDe is based on the concept of \textit{processes} which
\textit{``map input signals onto output signals''}.

Currently, the software implementation of ForSyDe is based upon the
Haskell programming language. The designer specifies the
system model in Haskell as a network of cooperating process
constructors with the assistance of the ForSyDe Library.

Until now, there has not been an automated way to synthesize ForSyDe
models (i.e.  generate an equivalent low-level implementation from
which to build real hardware) .  However, as a result of this thesis,
hardware synthesis is now a feature of ForSyDe, enabling ForSyDe
designs to finally reach silicon.  That is possible thanks to the
development of a ForSyDe-to-VHDL compiler.  By using this compiler, a
ForSyDe model can be first translated to synthesizable
VHDL93 (one of the two most common hardware design
languages) and then, the designer can use any of the existing
VHDL-tools to synthesize the model.

This thesis report is aimed at documenting the background, design,
implementation and use of the compiler.
