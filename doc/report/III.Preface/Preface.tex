This thesis report is aimed at documenting the background, design,
implementation and use of a compiler which translates ForSyDe (\textit{Formal
  System Development}) specifications into VHDL93 \cite{vhdl93} synthesizable
code.

The current implementation of ForSyDe, in which the compiler was embedded,
is written in the Haskell \cite{haskell} programming language. That
means the compiler has unavoidably been coded in the same
language. For that reason, \textbf{it will be assumed that the reader
of this report is fluent in Haskell}. Introducing the reader to
Haskell is out of the scope of the present text. There are many
resources from where to acquire the necessary knowledge. Just to
mention three of them, \textit{A Gentle Introduction to
Haskell} \cite{haskellgentle} is available online at no cost and is
supposed to be a friendly text for the newcomers, Thomson's book
\textit{Haskell: the Craft of Functional Programming} \cite{craft} on
the other hand is longer but covers the language in greater detail and
finally the upcoming \textit{Real-World Haskell} book\footnote{Not
ready by the time of writing this preface.} \cite{realworldhaskell}
will be freely available online and will help to get involved with
serious real-world Haskell code and practical Haskell
programming. Mastering Haskell is not a prerequisite to understand the
overall content of the thesis. However, wide Haskell
programming-experience and familiarity with its common extensions
would definitively be useful (and is probably needed) to comprehend
the compiler's design and implementation details.

In addition, it will be assumed that the reader is familiar with
digital hardware design and development through HDLs (\textit{Hardware
Design Languages}) or, in the worst case, understands the concepts
behind it. This prerequisite is weaker than knowing Haskell, since
specific knowledge of VHDL is not essential to read the
report. However, being aware of the purpose of an HDL is vital to
understand the goals which were achieved.

This thesis report is divided in five chapters:

\textbf{Chapter \ref{chap:intro}} introduces the reader to ForSyDe and
the thesis goals. It also introduces the concept of Embedded DSL
(\textit{Domain Specific Language}) which is essential to understand
ForSyDe's implementation.

\textbf{Chapter \ref{chap:vs}} is targeted at comparing ForSyDe with
Lava, a successful Haskell-embedded HDL (\textit{Hardware Description
Language} ) and verification environment. The comparison was written
before developing the compiler, in order to acquire the necessary
background in the \textit{Hardware Design and Functional Languages}
research field, hoping to be able to later reuse previous research
results and provide ForSyDe's compiler with state-of-the-art features.

ForSyDe's translator to VHDL turned out to be strongly influenced by
Lava. As a result of the comparison, the compiler is intended to
inherit Lava's virtues and overcome some of its problems such as its
current lack of component reusability on the
compiler-level. \textbf{Chapter \ref{chap:design}} describes the
design of the compiler and the motivation behind it whereas
\textbf{Chapter \ref{chap:user}} is targeted at the end-user and
contains a tutorial to help getting familiar with the tool and its
API.

Finally \textbf{Chapter \ref{chap:conc}} closes the thesis analyzing
its results and outlining potential improvements and further work.

As an add-on, \textbf{appendix \ref{chap:hacker}} has been written with future
developers in mind. They will surely find this appendix useful to get
familiar with the compiler's implementation in first instance, and
later improve it or extend it at will (the sources are available under
the BSD licence at
\url{http://www.imit.kth.se/info/FOFU/ForSyDe/HDForSyDe/}).
