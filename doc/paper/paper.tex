\documentclass[preprint,natbib]{sigplanconf}

%\usepackage[T1]{fontenc}
\usepackage{helvet}
\usepackage{amsmath}
\usepackage{natbib}
\usepackage{color}
\usepackage{graphicx}
\bibpunct();A{},
\let\cite=\citep


\graphicspath{{figures/}}

% Comments (1: Comments visible in text; 0: Comments not visible in text)
\def\commandmode{1}

\ifnum \commandmode=1
  \newcommand{\ingo}[1]{\smallskip \textbf{Comment Ingo:} \textsl{#1} \smallskip}
  \newcommand{\alfonso}[1]{\smallskip \textbf{Comment Alfonso:} \textsl{#1} \smallskip}
\else
  \newcommand{\ingo}[1]{}
  \newcommand{\alfonso}[1]{}
\fi

\begin{document}

\conferenceinfo{ICFP'08}{September 22--24, 2008, Victoria, British Columbia, Canada} 
\copyrightyear{2008} 
%\copyrightdata{[to be supplied]} 

\titlebanner{DRAFT---Do not distribute}        % These are ignored unless
\preprintfooter{}                                             % 'preprint' option specified.

\title{Compiling ForSyDe}
\subtitle{Deep-embedding a behavioural and polymorphic DSL in Haskell}

\authorinfo{Alfonso Acosta\and Ingo Sander}
           {Royal Institute of Technology, Stockholm, Sweden}
           {\{alfonsoa,ingo\}@kth.se}

\maketitle

\begin{abstract}
This paper introduces a general technique based on dynamic types and Template Haskell to deep-embed DSLs in Haskell without losing the power of polymorhic types and to describe computations in plain Haskell. We successfully applied the technique for the analysis of systems in the ForSyDe design methodology. ForSyDe system models can be described using any Haskell type for signal values and any Haskell expression to describe computations. Depending on the restrictions of the analyzer backend, the backend can either support the full language (simulation) or a restricted subset of the language (synthesis to VHDL). 

\end{abstract}

\category{D.3.2}{Programming Languages}{Language Classifications}[Haskell]
\category{D.3.4}{Programming Languages}{Processors}[Compilers]
\category{C.0}{General}{}[Systems specification methodology]
\category{D.3.2}{Programming Languages}{Language Classifications}[Design languages]
\category{B.5.2}{Register-Transfer-Level Implementation}{Design Aids}[Hardware description languages]

\terms
term1, term2

\keywords
keyword1, keyword2

\section{Introduction}
\label{sec:Introduction}

% Intro ForSyDe modeling language 
ForSyDe is a design methodology that has been designed to raise the abstraction level of system design. It uses a behavioral system domain specific language (DSL) with polymorphic signal values, which means that signal values can be of any Haskell type.
 
% Objectives
The initial implementation of the modeling environment was shallow embedded in Haskell, which restricted its use to simulation and did not allow to analyse the systems model. 

% Solution
This paper introduces a general technique based on dynamic types and Template
Haskell \cite{} to deep embed DSLs in Haskell without losing the power
of polymorhic types (values in all deep embedded DSLs so far are
monomorphic (check!!!)) and to describe computations in plain Haskell
(no need to create an specific DSL for that). We successfully applied
the technique for the analysis ForSyDe of systems. This means that
system models can be described using any Haskell type for signal
values and any Haskell expression to describe computations. Depending
on the restrictions of the analyzer backend, the backend can either
support the full language (simulation) or a restricted subset of the
language (Synthesis to VHDL). 

% Outcome
% -- Compiler
% -- Simulation
% -- Graphical Backend

ForSyDe has been design as a system design methodology for heterogeneous embedded systems \cite{SanJan2004a}. A ForSyDe system model is a hierarchical concurrent process network, where processes communicate via signals. ForSyDe supports different models of computation (MoCs), which are connected via domain interfaces. The ForSyDe library comprises modeling elements for the different models of computation and is written in Haskell. 

%This paper presents an embedded hardware compiler for the synchronous MoC of ForSyDe. A signal is defined as a sequence of events, where each event has a tag and a value. Although ForSyDe supports several models of computation \cite{}, we focus in this paper only on the synchronous model of computation, since 



\cite{BjeCla1998}


\section{The Problem}
\subsection{Why the shallow-embedded implementation is not enough}


\subsection{Sumarizing the problem}
\alfonso{We should further elaborate on this, but the explanation must
  cover these points }

The analysis capabilities of ForSyDe's initial implementation, a
shallow-embedded Haskell DSL, were sadly reduced to sequential simulation.

Being able to analyze and process the structure of system models is a
basic need for any system design environment. Thus, the ForSyDe
methology required a new tool, which should cumply with the following
requisites:

\begin{itemize}
\item Natively simulate and analyze the structure of systems. As it
  will latter be seen, here are a few examples of what is offered by
  our final solution:
  \begin{itemize}
    \item Sequential simulation (the only feature included in
      ForSyDe's early implementation).
    \item Compilation to VHDL.
    \item Integration with third party EDA tools with which to
      simulate, verify and synthesize the generated VHDL code.
    \item Compilation to a graphical format (GraphML).
  \end{itemize}
\item The new tool should be as faithful with the original API as
  possible, maintaining the same recursive-equation style.
\item As it happened before, process constructors must be
  polymorphic and systems should be able to use signals of
  of any Haskell type.
\item The new tool should also preserve the ability to express 
  process computations in plain standard Haskell, without
  restrictions or the need of a DSL specifically designed for that purpose.
\item The simulation backend must fully support every possible system
  model. However, there are other backends, such as hardware
  synthesis, in which certain Haskell types and expressions are
  meanigless or difficult to support (e.g. lists and list
  comprehensions). Thus, the tool can be more restrictive based on the
  analysis performed. Depending on such analysis, system models might
  be forced to comply with a set of primitive signal types, however, that
  set must alwasy contain:
  \begin{itemize}
  \item User-defined enumerated types
    (i.e. Algebraic types of kind one, whose data constructors have all
    zero arity)
  \item Container types, nested as deeply as it is
    required: fixed-sized vectors, tuples of arbitrary
    size\footnote{upper-bounded by the maximum tuple size} and
    \texttt{AbstExt} values.
  \end{itemize}
  

\end{itemize}

\section{Our solution}
\alfonso{TODO: introduction}
\subsection{Template Haskell}
\alfonso{TODO: introduction to template haskell}
\subsection{\texttt{Data} and \texttt{Typeable}}
\alfonso{TODO: introduction to the Data and Typeable classes}

\input{4.TheDetails/TheDetails}
\input{5.OtherProblems/OtherProblems}
\input{6.RelatedWork/RelatedWork}
\input{7.Conclusions/Conclusions}


\appendix

\section{Appendix Title}

This is the text of the appendix, if you need one.

\acks

Acknowledgments, if needed.

\bibliography{paper,ref_ingo}
\bibliographystyle{plainnat}


\end{document}

